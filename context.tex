\section{The Oceanographic Context}
\label{sec:context}

The vastness of the ocean and the lack of sustained human presence in
most of its surface, cuopled with the harshness of the environment
whether it be the open ocean or coastal zones, mandates the use of
multiple sensing and perception modalities. Using multiple platforms
allows for observing larger spatial contexts; if these platforms could
be sustained then both spatio-temporal measurements can then aid in
bio-geophysical process studies of the upper ocean as well as in
security focused needs. However, the cost imposed by manned platforms
is prohibitive\footnote{Typical day costs of a research vessel amount
  to about \$22,000.}, and sustained human presence on such vessels is
unrealistic in both human and financial costs. Satellite remote
sensing using a train of spacecraft is a possibility, even if image
resolution is often distorted by atmospheric effects, and the lack of
pentration of active optical or radar based measurements beyond a
meter of the upper water column. 

Unmanned platforms, such as UAVs, ASVs, and AUVs are a viable
alternative which cost-permitting can be deployed across a large area
simultaneously while making measurements across space and time. For
oceanographic measurements, the meso-scale ($\sim 50$ Km$^2$) is
considered a viable spatial region of study and could also be
considered a starting point for observing and monitoring a region for
security. For instance, frontal zones, where two disparate bodies of
water come together to form a surface expression, are important
biological phenomenon of study. Typically, they can occur at a range
of spatial scales, from several hundred meters up to many thousand
kilometers \cite{belkin2007fronts} and are a feature which can
catalyze the generation of mesoscale ($\sim 100$ Km) meanders,
‘eddies’ and rings, and sub-mesoscale ($\sim 10$ Km) 'filaments' other
smaller structures all of which have substantial impact on ocean
variability and bio-geochemistry. Exploration of such dynamic features
with a single platform will unlikely provide an understanding of its
variability over space and time, while multiple robotic platforms, can
provide not only measurements spread spatially, but an understanding
of their temporal variability \cite{pinto20,pinto22}.

Marine robotic platforms are increasingly cost-effective, can stay 'on
station' for periods of sustained time, and can do so in unseasonably
harsh conditions, in comparison to manned platforms. Their cost and
high-resolution data gathering capabilities implies that a number of
these assets could be deployed simultaneously for targeted missions
for dual-use. To do so, requires a judicious mix of assets with
different capabilities and operating conditions leading to information
derived from multiple sources with different levels of synopticity to
provide situational awareness to either an oceanographer or a
warfighter.
