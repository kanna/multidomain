\section{The Oceanographic Context}
\label{sec:context}

The vastness of the ocean and the lack of sustained human presence in
most of its surface, cuopled with the harshness of the environment
whether it be the open ocean or coastal zones, mandates the use of
multiple sensing and perception modalities. Using multiple platforms
allows for observing larger spatial contexts; if these platforms could
be sustained then both spatio-temporal measurements can then aid in
bio-geophysical process studies of the upper ocean as well as in
security focused needs. However, the cost imposed by manned platforms
is prohibitive\footnote{Typical day costs of a research vessel amount
  to about \$22,000.}, and sustained human presence on such vessels is
unrealistic in both human and financial costs. Satellite remote
sensing using a train of spacecraft is a possibility, even if image
resolution is often distorted by atmospheric effects, and the lack of
pentration of active optical or radar based measurements beyond a
meter of the upper water column.

For the most part, scientists use research vessels to make discrete
water sample measurements by repeatedly stopping the vessel and lowering
instruments for measurements and sample collection. These measurements
are pulled together to discern what are continuous processes in space
and time. Underway measurements including those towed behind the
research vessel such as toyo's (CITE) provide continuous measurements
but are constrained by the vessel's movement and have no independant
means to adapt to variability in the water-column. Eulerian and
Lagrangian approaches to observations such as buoys and \texttt{ARGO}
floats \cite{roemmich09} can make accurate high-resolution assessments
over space and time, but are constrained by what water mass passes by or
the impact of currents on the device.

Even if such approaches provide us a window into the workings of dynamic
processes, they are constrained in neglecting higher orders of
variability. As our knowledge of ocean processes has progressed, we have
learned that the more we observe, the more important the small scale
features are to understanding ocean dynamics. Equally that they don't
linearly scale up to add to the large patterns we have used to describe
the inner workings of our oceans, as in the past. 

Recent developments in mobile marine robots, sensing and Artificial
Intelligence (AI) have been slowly making an impact and dealing with
scales (micro to macro) of observation and space-time aliasing (cite) to
provide measurements at higher spectral, temporal and spatial
scales. Not only do such vehicles make continuous measurements, but
their onboard computational capabilities allow for embedded
decision-making to make observations at the `right place and time'. Yet
robots in singleton traversing the water-column still face the challenge
of dealing with the reality of the large scale of the ocean, starting
with meso-scale ($\sim 50$ Km$^2$) observations. This has led to the
belief that networked systems, including the Internet-of-Things (IoT),
could provide a wider and deeper view into our oceans. So in theory
while tens of hundreds of inexpensive robots can and have been proposed,
the practical notion of deploying more than a few devices at a time, to
survive the harsh confines of the ocean while returning needed
observational data has proven to be impractical.  Costs associated with
hardware might appear to be the primary bottleneck; however the primary
issue relates to the placement, data fusion, coordination and control of
multiple devices and therefore a constraint driven by software. Amongst
these challenges, the inter-networking of devices communicating
seamlessly and cost-effectively provides the fundamental 'glue' while
providing situational awareness aiding observational coordination and
logistical coordination (e.g. for launch and recovery).

Reliable communications becomes more relevant in remote operations due
to the inherent difficulty in physically accessing system state since
these operations feature opportunist and sporadic connection windows.
Networked infrastructure on vehicles is an important way forward to
alleviating this challenge in coordination which in turn could drive
the entire software stack onboard and on-shore and to form the basis
for sensing, data collection and assimilation at scale. 

Networking infrastructure in of itself in the maritime environment has
typically related to providing internet services for vessels or oil and
gas platforms \cite{jiang19,yau19,yau20} and less so with scientific
exploration. While satellite communications (SatComms) have focused on
such services out at sea, they have been expensive to use over sustained
periods of time for scientific exploration.  Networked infrastructure
can vastly aid in ocean exploration, permit a view of the ocean across
synoptic scales and with multi-domain platforms in the air, on the
surface and underwater, make a case for coordinated observations of the
worlds oceans, and how such exploration will likely impact the future of
ocean science. 

Unmanned platforms, such as UAVs, ASVs, and AUVs are a viable
alternative which cost-permitting can be deployed across a large area
simultaneously while making measurements across space and time. For
oceanographic measurements, the meso-scale is considered a viable
spatial region of study and could also be considered a starting point
for observing and monitoring a region for security. For instance,
frontal zones, where two disparate bodies of water come together to form
a surface expression, are important biological phenomenon of
study. Typically, they can occur at a range of spatial scales, from
several hundred meters up to many thousand kilometers
\cite{belkin2007fronts} and are a feature which can catalyze the
generation of mesoscale ($\sim 100$ Km) meanders, ‘eddies’ and rings,
and sub-mesoscale ($\sim 10$ Km) 'filaments' other smaller structures
all of which have substantial impact on ocean variability and
bio-geochemistry. Exploration of such dynamic features with a single
platform will unlikely provide an understanding of its variability over
space and time, while multiple robotic platforms, can provide not only
measurements spread spatially, but an understanding of their temporal
variability \cite{pinto20,pinto22}.

Spatial resolution of data is another principal challenge in ocean
observation. Typically, measurements made using manned vessels involving
repeated stop-and-go of the research vessel, to deploy sensors and
obtain samples with a winch, then moving the ship to the next station
after recovering the sensors only to repeat the process. Doing so over
large spatial scales essentially ``smears'' the time across these
measurements so aliasing \cite{} in this manner removes the correlation
between one measurement with another, unless they're made in some
proximity. The nature and environmental conditions contribute to how
correlated such measurements could be (or not) and as a result; these
are a manifestation of such oceanographic point-measurements across
large spatial and temporal scales. Conversely, untethered robotic
vehicles while moving slowly through the water column, are making
continuous measurements in high-resolution. As a result the principal
challenge is ensuring there is adequate onboard energy available for
operations in the order of a few days, depending on the power
source. Most platforms carry batteries as their primary source; recent
ASV's commercially available however leverage wind, solar or wave energy
(or a mixture of these) and as a result can sustain themselves far
longer and have proven themselves in the harshest of conditions, for
data gathering and observation. Unpowered AUV's or gliders are even more
effective in this context, since they leverage their kinetic energy of
movement through the water column, combined with a ballasting process
with a buoyancy engine to change their potential energy, resulting in
months-long missions collecting data in a saw-tooth (or a 'Yo-Yo')
pattern.

