\section*{Abstract}


\section*{Introduction}

Coastal zones provide the interface between the terrestrial domain and
the large expanse of the ocean which cover over 70\% of the surface of
the Earth. The transition between land and the ocean allows a range of
activities that make this domain critical for human wellbeing; from
trade, to tourism and sport, to its impact on food security for
sustainance based fishing communities as well as on national and
homeland security for facilities. With more than 600 Million people
living within 10 km of a coastline in the world, the impact of the
changing oceans has a disproportionate effect on a large area and
populations. 

While our understanding of such coastal environments have improved by
increased measurements of key variables especially sea surface height
(SSH), sea surface temperature (SST), tides, and pressure, all of
which have allowed for increased predictive power of oceanographic
events such as tsunamis, coastal surges, harmful algal blooms, hypoxia
(oxygen depletion), oil spills and coastal pollution. Modeling the
coastal zone with higher resolution and predictive power, has been
notoriously difficult given the distinct lack of scientific knowledge
we have of these highly mixed regions. The interaction of bathymetry,
physical forcings, riverine/estuarine interactions with the mixed surf
zone with strong tidal influences, implies that predictive
capabilities especially for loading/retrieving warfighters and assets,
are hamstrung by limited knowledge. The non-linearity of air/sea/land
interactions, the infusion of fresh water from riverine sources, the
proximity of anthropgenic sources of pollution in addition to the
nominal complexity of hydro-dynamic flow add to the challenges in
observing, measuring and predicting the state of any coastal
zone. Another principal challenge is gathering data in a synoptic as
well as systematic manner, across cohesive spatial and temporal scales
which can aid in the understanding of bio-geophysical
processes. Amalgamating, integrating and assimilating such data across
a range of critical ocean variables then, becomes a principal
technical task that to date have been difficult. With the onset of the
robotics and sensor technology revolution over the past decade, along
with the onset of explosion in work in both Machine Learning and
Artificial Intelligence methodologies, some of these challenges can
now be addressed.

Over the last decade a range of efforts primarily at universities in
the US, Europe and Australia have led to the development of low-cost
marine robotic platforms \cite{}. These have benefited from
incremental developments in autonomous low-level control \cite{} as
well as in hardware subsystems and payloads related to measuring some
of the essential variables noted above. Key variables in the ocean are
often not directly measurable; however sensing technology exists which
can be a proxy. Coupled with smaller form factor and optimized energy
use and leveraging battery technologies, these platforms have had
extended use in the aerial, surface and underwater domains. For
instance, one popular low-cost autonomous underwater vehicle (AUVs)
now has the capability to be on station for 5 days \cite{}. Small
unmanned aerial vehicles (UAVs) have had increased use in the battle
space on land; however those same technologies have proven to be
equally useful over the water, both as fixed wing as well as rotary
platforms. Autonomous surface vehicles (ASVs) however have made the
biggest strides despite being in the harsh surface environment, in
part because they have been able to capitalize on renewable sources of
energy (wind, waves and solar) to become persistent long-range assets
for dual use capabilities \cite{}. Finally, the advent of Small
Satellite platforms initially built by students for single purpose
uses, have prolifirated with significant advances in payloads, control
and propulsion leading to their adoption commercially for a range of
earth observing capabilities including for ocean observation \cite{}.

Predictive models come in a wide range. 

Agile low‐cost Small Satellite platforms, coupled with aerial, surface
and underwater mobile and immobile platforms can provide low cost
high‐resolution data at scales which are yet to be realized and
unavailable to date
