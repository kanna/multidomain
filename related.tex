\section{Related Work}
\label{sec:related}


The quest for a deeper understanding of the oceans through interdisciplinary, science-driven robotic exploration has seen promising advances in recent years. While the deployment of multiple and heterogeneous robotic vehicles for scientific exploration remains relatively novel, there's an anticipation of significant growth in this domain in the near future (de Sousa, 2021). Several efforts in this direction provide context for the work presented:

Robotic Asset Sharing: Initiatives such as Oceanids C2 demonstrate the potential of sharing live data feeds from robotic deployments over the internet. However, these endeavors are primarily geared towards data dissemination and lack the robustness required for real-time exploration and analysis (Harris et al., 2020).

Coastal Front Tracking: McCammon et al.'s research offered a glimpse into the potential of autonomous vehicles in tracking coastal salinity fronts. While their centralized planning approach provided valuable insights, the methodology deployed in the current study extends this to the open ocean, emphasizing real-time adaptation and a richer software infrastructure (Das et al., 2011; Faria et al., 2014b; de Sousa et al., 2016a; Py et al., 2016; de Sousa et al., 2016b; Chrpa et al., 2017; Fossum et al., 2018; Ferreira et al., 2019; Costa et al., 2018; Dias et al., 2020).

Seabed Mapping with Ocean Infinity: The Ocean Infinity campaigns, which employ a fleet of AUVs paired with ASVs and a manned vessel, showcase the potential of coordinated ocean exploration. However, their focus on the seabed and static behaviors distinguishes their approach from the dynamic exploration of the upper water column pursued in the current study (Rumson, 2018).

Traditional AUV Deployments: While AUVs, especially in the coastal regions, have seen more widespread adoption (Schmidt et al., 1996; Gottlieb et al., 2012; Smith et al., 2011, 2014; Das et al., 2015; Fossum et al., 2019; McCammon et al., 2021), the coordinated deployment of aerial, surface, and underwater vehicles in the harsher open ocean environment presents a unique challenge and innovation.

Challenges in Multi-Vehicle Coordination: Historically, multi-vehicle deployments from various institutions have been hampered by system incompatibilities, leading to their separate operation. The current work stands out with its integrated approach to networked control from a singular laboratory.

In sum, while strides have been made in oceanic robotic exploration, the integration of artificial intelligence, marine robotics, and comprehensive software infrastructure, as presented in this study, pushes the boundaries of what's possible in the domain.
