\section{Dual use applications}
\label{sec:applications}
%%Josh to use 9.8.23
%Attempt 1
The cutting-edge advancements in ocean robotic technologies, as epitomized by projects like METEOR, offer potential applications that extend far beyond the realm of scientific exploration and observation. One of the most pressing needs in today's fast-paced world is real-time environmental monitoring and disaster response. The capacity to swiftly deploy sensors and autonomous vehicles is invaluable during environmental crises. Whether we're dealing with oil spills, toxic algae blooms, radiation leaks, or the aftermath of natural disasters like tsunamis, these robotic systems can provide real-time assessments. This not only allows for rapid containment and mitigation actions but also aids in recovery operations by assessing underwater infrastructure damages.

In the context of maritime security and surveillance, defense agencies can significantly benefit from these technologies. The vast oceans present a challenging landscape for maintaining security. However, with autonomous platforms patrolling maritime borders, detecting unauthorized underwater activities, or monitoring critical infrastructures such as offshore oil platforms becomes feasible. This heightened security can deter potential threats, ensuring safer maritime operations.

The oceans are also a source of sustenance for many. Overfishing poses a significant threat to marine biodiversity. Using ocean robotic systems, authorities can monitor fish populations in real-time, ensuring that fishing activities remain within sustainable limits. This technology can also detect and report illegal fishing activities, ensuring that marine life is preserved for future generations.

As terrestrial resources become scarce, industries are increasingly looking towards the oceans. Deep-sea mining and exploration have the potential to yield significant economic benefits. Here, autonomous vehicles equipped with high-resolution sensors can play a crucial role in identifying mineral-rich sites and assessing the feasibility of mining operations.

Search and rescue operations in the vast oceans present a formidable challenge. Time is of the essence when it comes to saving lives. Rapidly deployable autonomous systems can scan extensive areas of the ocean, providing real-time data to rescue teams, increasing the likelihood of successful missions. Similarly, in the field of marine archaeology, these robotic systems can unveil the mysteries of the deep, revealing sunken cities, shipwrecks, and other historical artifacts without risking human lives.

Communication in areas with limited satellite connectivity is another challenge that these technologies can address. Autonomous surface vehicles equipped with communication tools can act as relay points, ensuring consistent communication channels.

A noteworthy consideration, especially in the realm of security and defense, is the potential for adversarial machine learning. As these robotic systems often rely on AI and machine learning models for decision-making and pattern recognition, they can be susceptible to adversarial attacks. These attacks manipulate input data to deceive the model, leading to incorrect outcomes. For security-centric applications, ensuring the robustness of these models against adversarial attacks is paramount.

In conclusion, while the primary motivation behind projects like METEOR is scientific exploration, the resultant technologies have a vast spectrum of dual-use applications. Their potential impact across various sectors will only grow as these systems become more sophisticated and affordable.